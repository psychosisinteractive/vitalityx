V\+Lib is the current Vitality driver system. You manipulate it using Interrupt 3\+Fh (refer to VitalityX Interrupt 3\+Fh (63), or \hyperlink{a00008}{S\+O\+F\+T\+I\+N\+T.\+md})

VitalityX takes care of initalizing your V\+Lib entry for you via Int 3fh\+:Eax=03h, which requires the following\+:


\begin{DoxyItemize}
\item E\+DI -\/ Pointer to the Function
\item E\+DX -\/ Mode
\end{DoxyItemize}

and it will set


\begin{DoxyItemize}
\item E\+BX -\/ Entry ID
\end{DoxyItemize}

Entry ID\textquotesingle{}s are how you call the driver functions. In order to call a function you must use Int 3fh\+:Eax=05h, which requires the following\+:


\begin{DoxyItemize}
\item E\+BX -\/ Entry ID
\end{DoxyItemize}

You\textquotesingle{}d notice that Int 3fh\+:Eax=03h outputs to E\+BX, and 3fh\+:Eax=05h takes in E\+BX and that is intentional. If you want to test your V\+Lib before running it then it would be as simple as 
\begin{DoxyCode}
; create vlib here
mov eax,03h
int 3fh ; moves the entry id to ebx
mov eax,05h
int 3fh ; calls the entry id, which should already have been set by the previous int 3fh
\end{DoxyCode}


In order to implement a V\+Lib function, a struct is passed defining the registers where the V\+Lib started. The struct is as defined below\+:


\begin{DoxyCode}
typedef struct \{
    uint32\_t eax, ebx, ecx, edx, esi, edi, esp, ebp, eip, eflags, cr3;
\} Registers;
\end{DoxyCode}


Thus, youd define a V\+Lib function in C as follows\+: 
\begin{DoxyCode}
void vlib\_function(Registers reg);
\end{DoxyCode}
 